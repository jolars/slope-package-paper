\PassOptionsToPackage{unicode,pdfusetitle}{hyperref}
\PassOptionsToPackage{hyphens}{xurl}
\PassOptionsToPackage{usenames,dvipsnames}{xcolor}

\documentclass[english,a4paper,14pt]{letter}

\usepackage{babel}

% Font settings
\usepackage[T1]{fontenc}
\usepackage{libertine}
\usepackage{inconsolata}

\usepackage{textcomp}
\usepackage{microtype}
\UseMicrotypeSet[protrusion]{basicmath}

\usepackage{xcolor}

% tables and lists
\usepackage{enumitem}
\usepackage{booktabs}

\usepackage{graphicx}

\usepackage{xurl}

\usepackage{fancyhdr}
\fancyhead{}
\fancyhead[L]{Larsson, Bogdan, Grzesiak, Massias, Wallin}
\fancyhead[R]{Cover Letter}

\pagestyle{fancy}
\thispagestyle{plain}

\usepackage{hyperref}
\hypersetup{
  pdftitle   = {Cover Letter},
  pdfauthor  = {Johan Larsson, Malgorzata Bogdan, Krystyna Grzesiak, Mathurin Massias, Jonas Wallin},
  colorlinks = true,
  linkcolor  = MidnightBlue,
  citecolor  = MidnightBlue,
  urlcolor   = MidnightBlue,
  breaklinks = true
}

\begin{document}

\begin{letter}{Journal of Statistical Software}
  \opening{Dear Editor,}

  We are thankful for the feedback on our previous submission and apologize for the
  issues encountered in the submission. Below, we provide a point-by-point response to the
  comments raised and outline the changes made to address them.

  We would, however, like to take this opportunity to clear up what we think might
  be a misunderstanding regarding the software and its scope. SLOPE solves
  a type of regularized regression problem so it is not related to \emph{slopes}
  in the geometric sense. Furthermore, SLOPE is distinct from standard \(\ell_1\)-penalized
  regression (the lasso), so we do not see the relevance of a full background
  on packages that implement the lasso. We do, however, realize that we
  neglected to discuss existing SLOPE implementations, even if to explain
  that there are in fact no other specific SLOPE software packages, and have
  added such a discussion in the introduction.

  We hope that you will find our revised submission suitable for publication in
  the Journal of Statistical Software.

  Here are the detailed comments and our answers:

  \begin{quote}
    Journal of Statistical Software requires that the software
    usage is illustrated on simple problems run step-by-step in the
    manuscript, showing command lines and their results. This is currently
    done for the R package but we would find beneficial to have a similar
    code for the other two packages.
  \end{quote}

  Thanks for the suggestion! We have added simple step-by-step examples
  for both the Python and Julia packages in the appendix of the manuscript and
  in the replication materials.

  \begin{quote}
    The introduction must include a discussion on related software
    implementations available in all languages (e.g., software
    implementations related to l1 penalized linear regression for instance,
    or even slope problem estimation), highlighting the specific
    contribution of slope.
  \end{quote}

  We think there might be a misunderstanding here, as SLOPE is not related to
  slopes in the geometric sense, but rather a type of regularized regression
  problem. It is also distinct from standard \(\ell_1\)-penalized regression
  (the lasso), which means that it would not be meaningful to include a full
  background on packages that implement the lasso. We have, however, added a
  discussion of existing SLOPE implementations in the introduction to clarify
  this point.

  \begin{quote}
    None of the [package files] can be uncompressed (or used) using gunzip.
  \end{quote}

  Thanks for pointing this out! We have packaged the package files
  correctly now and verified that they can be decompressed without issues.

  \begin{quote}
    It means that, at least a summary method is missing.
  \end{quote}

  Thanks for the suggestion! We have added \verb|summary()| methods for
  both \verb|SLOPE| and \verb|TrainedSLOPE| in the R package.

  \begin{quote}
    Other methods might
    be missing that would avoid exposing the internal structure of objects
    in the replication material [...]
  \end{quote}

  We are not sure we agree that the fields exposed in the objects should be
  considered internal structure (private) nor that it's idiomatic in R
  to hide them behind accessor methods, so we have respectfully chosen not to
  implement such methods. In the case of the SLOPE patterns, this object is
  computed as part of the fitting procedure and can therefore not be (easily)
  externalized to a separate function.

  \begin{quote}
    Using \texttt{example("SLOPE");
      plot(fit)}
    results in an empty plot. It is probably not expected: Can the authors
    look into this?
  \end{quote}

  Thanks! This was indeed a bug. In the new version of the package, we have
  added an entirely different (Cleveland dot plot) visualization of the
  solution for single fits, which should resolve this bug.

  \begin{quote}
    Replication material should be cleaned and simplified a bit, in
    particular to remove hidden configuration files coming from github. Make
    sure to upload only files required to reproduce results of the
    manuscript and to adapt README files to avoid having to use the github
    repository in the instructions (you can of course mention that these
    files are duplicated in a github repository but explain how to run the
    replication code from the local directory).
  \end{quote}

  Thanks, we have cleaned up the replication material and removed unnecessary
  files. The README file has also been updated to ensure that it describes
  both how to run the code from the local directory.

  \begin{quote}
    Following the instructions to run the benchmark given in the README
    file of \verb|benchmark_slope| (after activating a virtual environment for
    Python), we obtained: [...]
  \end{quote}

  This is related to not having the conda environment activated. We have
  updated the README file to clarify this.

  \begin{quote}
    If conda has to be used, this should be mentioned in the installation
    instructions. The README file of the replication material indicates that
    this is the case: please make the two files consistent or remove the
    README files of subdirectories.
  \end{quote}

  As in the previous answer, we have updated the README file to clarify
  the use of conda environments. We have also removed redundant README files
  from the subdirectories.

  \begin{quote}
    Also provide a more detailed information
    about setting the conda environment for user not familiar with it.
  \end{quote}

  We have added more detailed instructions on setting up the conda
  environment in the README file.

  \begin{quote}
    Running benchopt run in \verb|benchmark_slope| results in errors due to
    missing dependencies (skglm for instance). It would be preferable to
    provide a \verb|requirements.txt| file allowing users to install all required
    dependencies without the need to search what they are.
  \end{quote}

  benchopt has built-in support for managing dependencies via
  the benchmark definition files, through which the required dependencies
  are declared and then automatically installed when \verb|benchopt install|
  is run. We have updated the README file to clarify this.

  \begin{quote}
    Trying to run the same benchmark, we also had: [...]
    Can you instruct how this should be solved?
  \end{quote}

  This too is related to not having the proper dependencies installed. The
  new instructions in the README file should help avoid this issue.

  \begin{quote}
    Could you please also provide scripts with minimal command lines to
    run the equivalent of \verb|example.R| and/or \verb|real-data.R| with Python and
    Julia?
  \end{quote}

  We have supplied \verb|example.py|, \verb|example.cpp|, and \verb|example.jl| scripts that
  replicate the functionality of \verb|example.R| in Python, C++, and Julia, respectively.

  \begin{quote}
    \verb|help(package = "SLOPE")| shows that not man page titles are not in
    title styles.
  \end{quote}

  We have converted all manual pages to use title case in their titles.

  \begin{quote}
    Probably, \verb|plot.TrainedSLOPE()| should be removed from "Other
    model-tuning".
  \end{quote}

  Thanks! We have removed the duplicate reference.

  \begin{quote}
    It seems that the output of \verb|cvSLOPE| only gives a way to obtain the
    optimal hyperparameters. In R programming, it is standard to also return
    the best model fit with these parameters (see e.g., \verb|?e1071::tune|). Can
    the authors think of an handy way for the user to either obtain this
    trained model from \verb|cvSLOPE| or use the output of \verb|cvSLOPE| in a method that
    would directly train the optimal model?
  \end{quote}

  Thanks for the suggestion! You're right that this was an omission. We have now updated
  \verb|cvSLOPE()| to (optionally and by default), return the fitted model on the full
  data set. We've also introduced a new function (and generic) \verb|refit()| that
  can be used to fit SLOPE onto new data using the parameters selected by
  \verb|cvSLOPE()|.

  \closing{Yours sincerely,\\[1ex]
    Johan Larsson, Ma\l{}gorzata Bogdan, Krystyna Grzesiak, Mathurin Massias, and Jonas Wallin
  }

  \encl{manuscript, software, replication materials, original cover letter}
\end{letter}

\end{document}

\usepackage{libertine}

\title{Cover Letter for Submission to JSS}

\begin{document}

Blabla

\end{document}

